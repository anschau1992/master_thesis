\documentclass{task_description}

\begin{document}

\thispagestyle{firstpage}
\vspace*{23mm}%
\hfill\parbox[t]{65mm}{

Andreas Schaufelb\"uhl\\
Binzm\"uhlestrasse 41\\
8050 Z\"urich\\[5mm]
Matrikel-Nr. 12-918-843\\
andreas.schaufelbuehl@uzh.com\\[15mm]
\today \\
}
\vspace*{5mm}
\subsection*{Bachelor's Thesis Specification}
%

\section*{Evolving mobile apps by Linking  user reviews  feedback to source code entities\hspace{1em}}


%
\subsection*{Introduction}
%

User feedback plays a paramount role in the development and maintenance of mobile applications. The experience an end-user has with the app is a key concern when creating and maintaining a successful product. Consequently, developer teams are interested in incorporating opinions and feedback of end-users during the evolution of their software \cite{Krusche2014,Vithani2014}.
However, existing app distribution platforms provide limited support for developers to systematically filter, aggregate and classify user feedback to derive requirements. Moreover, manually reading each user review to gather useful feedback is not feasible considering that popular apps receive hundreds of reviews every day\cite{PaganoRE2013,Licorish2015}. 

For this reason automated approaches have been proposed in literature with the aim of reducing the effort required for analyzing feedback contained in user reviews\cite{Ha2013, Oh2013,IacobMSR2013, Iacob2014,GalvisCarrenoICSE2013,GuzmanRE2014,MaalejRE2015,ChenICSE2014,Fu2013,Vu2015,PalombaICSME2015,Gu2015}, but most of them only perform an automatic classification (or prioritization) of user reviews according to specific topics (\eg bugs, features etc.).
In recent work Panichella {\em et al}\cite{PanichellaICSME2015}  introduced an approach, called ARdoc (App Reviews Development Oriented Classifier)\footnote{http://www.ifi.uzh.ch/seal/people/panichella/tools/ARdoc.html}, able to classify user reviews into four category: Information Giving, Information Seeking, Feature Request and Problem Discovery.
This already takes away a huge amount of work for the developer, as it's not necessary any more to go through all the reviews and categorize them in valuable feedback (for maintenance perspectives) or useless comments.
However, this one dimensional classification results in an insufficient leverage of the available review information, because, for example, having huge amount of reviews classified as \textit{feature requests} is of limited use to developers trying to distill actionable tasks from the feedback contained in such huge clusters of reviews.

\newpage

\subsection*{The goals of this bachelor's thesis}
We argue that complement topic extraction from reviews with the capability of cluster similar reviews (assigning them to a well defined task of maintenance/evolution) and
  determining the corresponding parts of source code that need to be maintained/changed (in according to the suggested change tasks),  will concretely help developers planning app improvements and meeting market requirements.
%
Thus, the hight level goals of this thesis is to defined a mechanism able to to cluster app reviews having similar \textit{intentions} and \textit{link} them to source code entities. 
The research questions that guide this thesis are: 

\begin{itemize}
\item \textit{RQ1: To what extend is it possible to cluster/group app reviews user feedback
   and link them to source code entities?}
\item \textit{RQ2: To what extend developers of mobile apps address users requests?}
         Specifically, starting from the general \textbf{\textit{RQ2}} we derive two further sub-research questions:
         \begin{enumerate}
	\item \textit{RQ2-a: What kinds of feedback are usually  addressed by developers?}
	\item \textit{RQ2-b: What kind of feedback are usually not addressed by developers?} 
	\end{enumerate}
\end{itemize}      
      

\subsection*{Task description}
%

The main tasks of this thesis are:

\begin{enumerate}
\item Read up on the current state of the art in industry and research in the areas relevant to the thesis, including AR-Miner. %TODO: add more areas
\item Write a software program able to gather user reviews and save them into a database.
\item Extend the program, so it is capable to link clustered reviews with the corresponding source code.
\item Writing an academic report summarizing the results from the work on items 1 to 3.
\end{enumerate}


\subsection*{Deliverables}
%

The thesis is expected to run in three phases. During the first phase, existing scientific literature and existing tools need to be surveyed in order to gain an understanding of the current state of the art. In this thesis, this will specifically require the student to determine the most fitting tools for the implementation of the database and the front-end tool. Afterwards the student will gain the technological knowledge of the chosen tools. The second phase of the thesis consists of the whole implementation of the application in code as a running program. The $2^{\mathit{nd}}$ month is focused on setting up a database, which gathers all the user reviews from the different app stores. Over the $3^{\mathit{rd}}$ and $4^{\mathit{th}}$ month, the student built a technical solution to link clustered user reviews with source code entities. Afterwards it will be scientifically evaluated within the $5^{\mathit{th}}$ month. In the last phase the academic report (bachelor's thesis) is written and delivered in the  $6^{\mathit{th}}$ month.\\

The major milestones of the project are as follows:

\begin{tabular}{lp{10cm}}
When & What \\
\hline\noalign{\smallskip}
$1^{\mathit{st}}$ month & State of the Art review is finished. Tools are picked and learned. \\
$2^{\mathit{nd}}$ month & Database is set up and collecting user reviews. \\
$4^{\mathit{rd}}$ month &  Implementation of the Linking is finished. \\
$5^{\mathit{th}}$month & The proof-of-concept implementation is finished.\\
$6^{\mathit{th}}$month & Thesis is written, proof-of-concept is fully functional and delivered.
\end{tabular}

\newpage

\subsection*{Provided resources}

There is no nedd of special provided resources to realise this thesis

\subsection*{General thesis guidelines}

The typical rules of academic work must be followed. In
\cite{Bernstein:8} Bernstein describes a number of guidelines which
must be followed. At the end of the thesis, a final report has to be
written. The report should clearly be organized, follow the usual academic
report structure, and has to be written in English using our
s.e.a.l. \LaTeX-template. As implementing software is also part of this thesis, state-of-the-art
design, coding, and documentation standards for the software have to be obeyed.

\subsection*{Advisors:}

\noindent\textbf{Professor}: \\
\noindent Prof. Dr. Harald C. Gall \\
\\
\noindent\textbf{Responsible assistant}: \\
\noindent Dr. Sebastiano Panichella \\

\vspace{2em}
\noindent\textbf{Signatures:}

\vspace{3\baselineskip}
\noindent Andreasl Schaufelb\"uhl\hfill Prof. Dr. Harald C. Gall
\clearpage
\bibliographystyle{abbrv}
\bibliography{refs}

\end{document}
